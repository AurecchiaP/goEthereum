\documentclass[10pt]{article}
%-------------------------------------------------------------------------------
% Packages
%-------------------------------------------------------------------------------
\usepackage{fancyhdr}
\usepackage{graphicx}
\usepackage{amsmath,amssymb,amsfonts}
\usepackage{amsthm}
\usepackage{mathrsfs}
\usepackage{fullpage}
\usepackage{enumerate}
\usepackage{color}
\usepackage{geometry}
\usepackage{lastpage}
\setlength{\parskip}{\medskipamount}
\setlength{\parindent}{0pt}
\usepackage[utf8]{inputenc}
\usepackage{graphicx}
\usepackage{xcolor}
\usepackage{amsmath}
\usepackage{titlesec}
\usepackage{tikz}
\usetikzlibrary{arrows,automata}
\usepackage{indentfirst}
\usepackage{listings}
\lstset{
  numbers=left,
  stepnumber=1,
  firstnumber=1,
  numberfirstline=true
}

\definecolor{labelcolor}{RGB}{100,0,0}
\definecolor{shadow}{RGB}{70,70,70}

%-------------------------------------------------------------------------------
% PAGE STYLING
%-------------------------------------------------------------------------------

% Subject
\title{\large{\textsc{Distributed Systems Project Report}}\\
      \huge{Go Ethereum}}
\date{}
\author{}
\geometry{hmargin=2.5cm,vmargin=3cm, headsep=25pt}
\setlength{\headheight}{15pt}
\lhead{\today}

% Assignment number
% \chead{Assignment 1}

% Author's name
\rhead{Paolo Aurecchia \& Irene Jacob}
\cfoot{\thepage \,\,of \pageref{LastPage}}
\renewcommand{\footrulewidth}{0.4pt}
\setlength{\footskip}{15pt}

% Exercise numbering
\titleformat*{\section}{\Large}
\renewcommand{\thesection}{\arabic{section}.}
% Question numbering
\titleformat*{\subsection}{\color{shadow}\normalsize}
\renewcommand{\thesubsection}{\arabic{section}.\arabic{subsection}.}
% sub-question numbering
\titleformat*{\subsubsection}{\color{shadow}\normalsize}
\renewcommand{\thesubsubsection}{ \,\roman{subsubsection}\,) }

% adds indentation
% \setlength\parindent{24pt}

%-------------------------------------------------------------------------------
% BEGIN DOCUMENT
%-------------------------------------------------------------------------------
\begin{document}
\maketitle
\vspace{-30pt}
\thispagestyle{fancy}
\pagestyle{fancy}
%-------------------------------------------------------------------------------
% Exercise 1
%-------------------------------------------------------------------------------
\section{Idea}
The main idea of this project is to implement the game Go using Ethereum's
BlockChain and smart contracts.
\subsection{What is Go?}
Go is a strategy board game that was first invented more than 2500 years ago
in China: it then became popular also in other countries such as Korea and Japan,
and it developed different rules over the course of time.
Nowadays Go is known and played world-wide, even though the majority of players
still reside in Asia. While it is an essentially simple game to learn, it is also
very difficult to master, as the tree of possible moves that the players can take
can quickly become huge.

\subsection{How does the game work?}
The game is played by two opponents, black and white, which take turns placing
tokens -- called ``stones'' -- on a 19 by 19 board. The black player plays first,
and both players are allowed to pass their turns whenever they want.
When the two players pass their turns consecutively, the game ends.

The objective of the game is to capture more area on the board than the opponent,
by surrounding it with our stones.  (concepts of liberty, chain?)

\section{Implementation}

\subsection{Client}
The application is built using Truffle, a development environment for Ethereum.
The interface is done using HTML5, CSS3 and Bootstrap, and JavaScript and jQuery
are used to manage the backend of the application. The interaction with the
smart contracts is done through Truffle's interface/API, which uses web3.js and
provides additional tools to simplify the communication with the contracts.

When the user runs the application, he can use the user interface to create games,
 join existing games created by other users, and he can play in the games that
 he is part of.

\subsection{Contracts}
The contracts are written using the language Solidity.

We have two contracts: {\ttfamily Go.sol} and {\ttfamily GoGame.sol}. {\ttfamily Go.sol}
is a contract that is always deployed: it is unique, and it acts as a ``Factory''
 for the go games: when the user wants to create a new game, a method of {\ttfamily Go.sol}
 will be called, that then creates a new instance of {\ttfamily GoGame.sol} and adds it
 to its list of existing games.

 {\ttfamily GoGame.sol} represents a single go game: it keeps track of the of
 the board, the players and the state of the game. When a player makes a move,
its validity is checked directly on the contract, so that we ensure the fairness
of the game. If the move turns out to be valid, Difficult to use with applications that require fast replies, as we have to wait for blocks to be mined
Difficult to debug
New technologies - lack of documentation and sometimes limited features
Code needs to be optimised in order to reduce gas pricethe state of the game gets
updated (we update the board accordingly and we change the turn of the player).

If instead of making a move the player passes the turn, it is again checked on the
contract if it is valid. If that is the case, we update the turn of the game,
and if it was the second consecutive pass, the game ends, in which case
the contract calculates who the winner is and stores that in the contract.

\section{Using BlockChain}

\subsection{Benefits}
The main benefit of developing such a game using smart contracts and BlockChain
is the possibility to ensure the fairness of the game, since the logic and the
validity of the moves is checked on the contract. This would not be possible if
we were to implement the game using a normal client-server approach.

Another advantage is that we do not depend on the availability and reliability
of a server, since the BlockChain is a distributed system.

\subsection{Challenges}
One of the drawbacks of working with BlockChain is that we cannot rely on the
timing of the transactions, as we only have a rough idea of when the blocks
will be mined, and even if we did know the exact time, it might simply be too slow
 to be useful for certain applications.

Also, writing smart contracts can be quite troublesome, as it is difficult to
debug applications, there is no simple way to understand what is happening
during the execution of the transactions, and the error messages can be quite
cryptic.

Most of the technologies used are recent and still in development, too: there is
not much documentation and some features might be lacking, which again doesn't
make the development easier.

Finally, smart contracts' transactions cost money to be executed, which means
that the code and the algorithms used have to be highly optimised to reduce the
costs, and when possible it should be considered to move parts of the logic of
the application outside of the contracts, to simplify the operations of the
transactions.

\end{document}
%-------------------------------------------------------------------------------
